% !TEX TS-program = xelatex
% !TEX encoding = UTF-8 Unicode
% !Mode:: "TeX:UTF-8"

\documentclass{resume}
\usepackage{zh_CN-Adobefonts_external} % Simplified Chinese Support using external fonts (./fonts/zh_CN-Adobe/)
% \usepackage{NotoSansSC_external}
% \usepackage{NotoSerifCJKsc_external}
% \usepackage{zh_CN-Adobefonts_internal} % Simplified Chinese Support using system fonts
\usepackage{linespacing_fix} % disable extra space before next section

\usepackage{xcolor}
\definecolor{CVBlue}{RGB}{23,110,191}
\usepackage[backend=bibtex]{biblatex}

\usepackage{tikz}
\usetikzlibrary{calc}

\usepackage{geometry} % 调整页边距
% \geometry{a4paper,top=1cm}
% \geometry{a4paper,scale=1}
\geometry{a4paper,left=1.5cm,right=1.5cm,top=0.5cm,bottom=1cm}

\begin{document}
\pagenumbering{gobble} % suppress 

%%%% 利用tikz来定位照片,部分招聘单位可能需要“以貌取人”
\begin{tikzpicture}[remember picture, overlay] 
  \node[anchor = north east] at ($(current page.north east)+(-1.5cm,-0.8cm)$) {\includegraphics[height=3cm]{avatar.jpg}};
\end{tikzpicture}%

%%%% 利用tikz来定位页脚栏,电子版简历使用,黑白纸质打印效果可能并不好。这里只在第一页显示,如果需要每页都有,页脚或者background中加入。
\begin{tikzpicture}[remember picture, overlay] 
  \node[anchor = south,fill=CVBlue,draw=none,minimum width=\paperwidth,minimum height=1.5em,align=center,font=\footnotesize,text=white] at ($(current page.south)$) {\faLinkedinSquare \ https://www.linkedin.com/in/ryushane \qquad \faGithub \ https://github.com/ryushane \qquad \faRssSquare \ https://ryushane.com};
\end{tikzpicture}%

\name{刘某某}

\basicInfo{
  \email{ryushane.net@gmail.com} \textperiodcentered\ 
  \phone{114-514-1919810} 
}



\section{\faGraduationCap\  教育背景}
\datedsubsection{\textbf{知春路航空职业学院}}{2020/09 -- 2023/06}
\textit{硕士研究生} \quad \textit{电子信息工程学院} \quad  \textit{信息与通信工程}
\begin{itemize}
  \item \textit{专业排名}:Top 30\%
  \item \textit{主修课程}:DSP体系结构,时频分析,检测估计与调制
  \item \textit{研究方向}:制作未来道具
\end{itemize}
\datedsubsection{\textbf{昌平航空职业学院}}{2016/09 -- 2020/06}
\textit{本科生}\ \quad \textit{电子信息工程学院} \quad \textit{电子信息工程专业}
\begin{itemize}
  \item \textit{专业排名}:Top 5\%
  \item \textit{主修课程}:C语言,计算机软件基础,通信原理,信号与系统,数字信号处理,数字电路
\end{itemize}



\section{\faUsers\ 实习/项目经历}

\datedsubsection{\textbf{给老黄设计芯片}}{2022/05 -- 2023/01}
\begin{itemize}
  \item 开发了xx工具,大大提高了团队开发效率。
  \item 搭建了很牛逼的平台,验证xx功能,总之很吊。
\end{itemize}

\datedsubsection{\textbf{面向xx目标炼丹}}{2021/10 -- 2022/11}
\begin{itemize}
  \item 研究目的:我为啥要研究这个。
  \item 研究方法:采集了xx数据,用了xx方法,实现了xx。
  \item 研究结果:实现了xx效果,比原来提高了114倍。
\end{itemize}

\datedsubsection{\textbf{双路语音同传的无线收发系统}}{2019/07}
\begin{itemize}
  \item 研究目的:打比赛,混点综合量化保研。
  \item 实现方法:大腿很牛。
  \item 实验结果:信号很纯真,来到北邮综测,初试和综测双最高分,但是没进国赛,作为一名理工男我觉得这太酷了,很符合我对电赛的幻想。
\end{itemize}

% \clearpage % 换页

\section{\faCogs\ 其他说明}
% increase linespacing [parsep=0.5ex]
\begin{itemize}[parsep=0.5ex]
  \item 编程能力: 
  \begin{itemize}
    \item 熟练使用Python / C / Matlab
    \item Linux用的挺溜
    \item 会写Verilog
  \end{itemize}
  \item 语言能力:
  \begin{itemize}
    \item 过了四六级
  \end{itemize}
\end{itemize}



\section{\faInfo\ 学生工作}
\begin{itemize}[parsep=0.5ex]
  \item 古典部副部长,于文化祭出版社刊《冰菓》,帮助2-F班推理电影结局,在文化祭中参加猜谜、料理等比赛,在校外参与新年参拜、生雏祭等活动。

\end{itemize}

\section{\faFile\ 获奖情况}
\datedline{校级三好学生}{2021/10}
\datedline{北京市优秀毕业生}{2020/06}
\datedline{全国大学生电子设计竞赛一等奖}{2019/09}

\clearpage

%%%% 利用tikz来定位照片,部分招聘单位可能需要“以貌取人”
\begin{tikzpicture}[remember picture, overlay] 
  \node[anchor = north east] at ($(current page.north east)+(-1.5cm,-0.6cm)$) {\includegraphics[height=3cm]{avatar}};
\end{tikzpicture}%

%%%% 利用tikz来定位页脚栏,电子版简历使用,黑白纸质打印效果可能并不好。这里只在第一页显示,如果需要每页都有,页脚或者background中加入。
\begin{tikzpicture}[remember picture, overlay] 
  \node[anchor = south,fill=CVBlue,draw=none,minimum width=\paperwidth,minimum height=1.5em,align=center,font=\footnotesize,text=white] at ($(current page.south)$) {\faLinkedinSquare \ https://www.linkedin.com/in/ryushane \qquad \faGithub \ https://github.com/Ryushane \qquad \faRssSquare \ https://ryushane.com};
\end{tikzpicture}%

\name{Ryushane}

\basicInfo{
  \email{ryushane.net@gmail.com} \textperiodcentered\ 
  \phone{114-514-1919810} 
}

\section{\faGraduationCap\  Education Background}
\datedsubsection{\textbf{BeiHang University (BUAA)}}{2020/09 -- 2023/06}
{\par \textit{Master}\ of Information and Communication Engineering \par}%
% \textit{Master} \hfill Information and Communication Engineering Dept
\begin{itemize}
  \item Research Topics: Optimization and acceleration of Convolutional neural network on FPGA
  \item \textit{\nth{1} Prize} of freshman merit scholarship
  \item Merit Student of BUAA
\end{itemize}
\datedsubsection{\textbf{BeiHang University (BUAA)}}{2016/09 -- 2020/06}
{\par \textit{Bachelor}\ of Electronics and Information Engineering \par}%
\begin{itemize}
  \item \textit{GPA}: 3.78/4.00, \textit{Ranking}: Top 5\%
  \item Beijing Outstanding Graduate(Top \%5), \textit{\nth{1} Prize} on NUEDC, \textit{\nth{1} Prize} on CUMCM
\end{itemize}

\section{\faUsers\ Experience}
\datedsubsection{\textbf{ASIC Design and Verification in NVIDIA}}{2022/05 -- 2023/01}
\begin{itemize}
  \item Purpose: Get a intership experience and hope to convert.
  \item Result: Develope a tools which is gorgeous for ASIC team and improve our development efficiency.
\end{itemize}

\datedsubsection{\textbf{xx-based Network}}{2021/10 -- 2022/11}
\begin{itemize}
  \item Purpose: I need to get a master degree.
  \item Method: Data collection, using xx method.
  \item Result: Achieving a 114514x speedup compared to Raspberry Pi 3b.
\end{itemize}

\datedsubsection{\textbf{Wireless transceiver system for two-way voice simultaneous transmission}}{2020/07 -- 2021/04}
\begin{itemize}
  \item Purpose: Get some comprehensive quantification for postgraduate recommendation.
  \item Method: Ride the dalao's coattails.
  \item Result: The signal is very good, came to BUPT for comprehensive test. Both preliminary examination and comprehensive test scores are highest. But we did not advance to the national competition. As a polytechnic man I think it's really cool and fits my fantasy about NUEDC.
\end{itemize}

\section{\faCogs\ Skills}
% increase linespacing [parsep=0.5ex]
\begin{itemize}[parsep=0.5ex]
  \item Proficient in using Verilog for FPGA development, familiar with the Vivado development environment.
  \item Proficient in Python / C programming, familiar with Tcl scripting and Linux development environment.
  \item CET-6
\end{itemize}

\section{\faInfo\ Campus Activities}
\begin{itemize}[parsep=0.5ex]
  \item Vice President of the Classical Department, publishes the magazine "Ice Tang" at the Cultural Festival, helps the 2-F class to deduce the ending of the movie, participates in the riddle-guessing and cooking competitions at the Cultural Festival, and participates in the New Year's Day worship and the Hatchling Festival outside of school.
\end{itemize}



\end{document}
